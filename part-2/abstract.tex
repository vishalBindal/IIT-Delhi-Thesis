\begin{abstract}
Ability to automatically detect and recover from failures is an important but challenging problem for an autonomous robot due to the presence of perceptual noise and uncertainty in discovering the goal state.
%
While existing approaches deal with this problem by training over data specifically annotated for failures, our goal in this work is to achieve this task, in a self-supervised manner, without requiring any such explicit annotation. 
%
Central to our approach is a neuro-symbolic object-centric representation of the state in the form of a dense scene graph. The core learning component of our approach consists of (a) neural module representing the transition function trained using data generated from an existing neuro-symbolic planner and (b) neural discriminators, trained in self-supervised manner, which not only detect failures, but also provide useful information about which objects in the scene need to be acted upon, localising the error and aiding an efficient discrepancy-aware synthesis of a recovery plan.  
%
Our technique is simple: (a) At any given execution step, we use a combination of our learned transition function and discriminator to estimate if a failure has occurred and also determine objects that need to be moved to recover from the failure (b) a recovery mechanism searches for an efficient plan to one of the states lying on the original plan %the state where failure had occurred 
using a forward search guided with learned heuristics.  
%
% When the last correct state may not be the closest to the failure state, we devise an anytime version of our algorithm to search for the optimal point on the original plan to the goal.
%
Experiments on a simulated data set with a variety of failures during plan execution shows the effectiveness of our approach compared to existing baselines.
\end{abstract}

% Two or three meaningful keywords should be added here
\keywords{Learning and Planning, Neuro-symbolic Models, Failure Recovery.} 

\iffalse
The purpose of this document is to provide both the basic paper template and submission guidelines. Abstracts should be a single paragraph, between 4--6 sentences long, ideally. Gross violations will trigger corrections at the camera-ready phase.
\fi