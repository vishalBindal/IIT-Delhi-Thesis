\begin{abstract}
Ability to automatically detect and recover from failures is an important but challenging problem for an autonomous robot due to the presence of perceptual noise and uncertainty in discovering the goal state. %This is a challenging task when dealing with real perceptual data due to the presence of noise, as well as uncertainty related to the goal state. 
While existing approaches deal with this problem by training over data specifically annotated for failures, our goal in this work is to achieve this task, in a self-supervised manner, without requiring any such explicit annotation. At the heart of our approach is a neuro-symbolic object-centric representation of the state in the form of a dense scene graph. The core learning component of our approach consists of (a) neural module representing the transition function trained using data generated from an existing neuro-symbolic planner and (b) neural discriminators, trained in self-supervised manner, which not only detect failures, but also provide useful information about which objects in the scene need to be acted upon, localising the error and resulting in an efficient discrepancy-aware recovery mechanism. Our technique is simple: (a) At any given execution step, we use a combination of our learned transition function and discriminator to flag if a failure has occurred and also identify objects that need to be moved to recover from the failure (b) recovery mechanism kicks in searching for an efficient path to the state where failure had occurred using a variation of heuristic-guided forward search. When the last correct state may not be the closest to the failure state, we devise an anytime version of our algorithm to search for the optimal point on the original path to the goal.
%path to the state where the failure occurred; search is made efficient by exploiting information about which objects need to be moved to reach the desired sub-goal. 
%correct statein wherein we 
%which exploits the knowledge about which objects need to be moved and 
%search for an efficient plan to reach the desired sub-goal, i.e., the last state where the error occurred. 
%We also experiment with a variation where we sub-goal to the closest point on the original path resulting in any time version of our algorithm. %Often, the state where error occurred may not be closest to the one obtained due to failure, and to handle such scenarios, we propose an anytime version of our approach which sub-goals to the closest point on the original path to the goal. 
Experiments on a simulated dataset with a variety of failures during plan execution shows the effectiveness of our approach compared to existing baselines.
\end{abstract}

% Two or three meaningful keywords should be added here
\keywords{Learning and planning, Neuro-symbolic models, Failure recovery.} 




\iffalse
The purpose of this document is to provide both the basic paper template and submission guidelines. Abstracts should be a single paragraph, between 4--6 sentences long, ideally. Gross violations will trigger corrections at the camera-ready phase.
\fi