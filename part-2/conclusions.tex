\section{Conclusion}\label{sec:conclusion}
We present a discrepancy-aware neuro-symbolic approach for plan recovery from failures. Unlike existing approaches, we do not require hand-annotated data of failures, rather we make use of self-supervision to train our recovery model. %, rather we make use of self-supervision to learn our underlying failure discovery and recovery mechanism. 
Our approach makes use of object-centric representation of the state in the form a dense scene-graph. We train neural modules to learn the transition function based on data gathered from an existing neuro-symbolic planner. Additionally, we train neural discriminators, trained via the help other states encountered during execution as negatives, to help us distinguish the representations of the simulated state (desired) from the failure states. Once a failure is detected, a recovery plan is constructed to join back the originally constructed plan at an appropriate point. 
%Directions for future work include working with partial observability during plan execution, and dealing with more complex failure scenarios such as objects dis-appearing from the scene, e.g., falling off the table.