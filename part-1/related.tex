\section{Related Works}

\textbf{Grounding Instructions to Robot Control.} Traditional approaches for grounding instructions assume a symbolic description of the environment state 
and robot actions. 
%
These approaches learn to associate phrases in an instruction with symbols conveying their meaning from 
annotated datasets. 
%
For example, \cite{howard2014natural,paul2016efficient,tellex2011approaching} map language to discrete motion constraints that can be provided to a motion planner for trajectory generation, \cite{matuszek2013learning,knepper2013ikeabot} parse language to a logical parse that includes  robot control actions and \cite{gopalan2018sequence,williams2018learning} use sequence to sequence modeling to ground language to LTL formulae capturing high-level task specifications. 
%
% Such approaches assume a symbolic model of the world (either hand coded or a-priori learned). 
% The approach presented in this work, additionally learns dense disentangled representations for 
% spatial concepts as well as actions from data while learning a grounding for an input instruction.  
%
In contrast to such approaches, the proposed work doesn't assume a symbolic model of the world and learns spatial and visual concepts and action semantics from data. 

\textbf{Inferring Plans from Instructions.} Another set of approaches focus on multi-stage tasks (e.g., cooking, assembly etc.) and propose 
learners that can infer action sequences by observing human task demonstrations. 
%
The work in ~\cite{paxton2019prospection} proposes a recurrent neural architecture that converts a natural language instruction to a sequence of intermediate goals for robot execution. 
%
Efforts such as \cite{shah2018bayesian,wang2020learning,kress2008translating} learn LTL task specifications from humans demonstrations and introduce a model for effectively searching in the space of programs. 
%
In another effort, ~\cite{lazaro2019beyond}, authors introduce a cognitively-inspired approach that infers 
a likely program from a symbolic generative grammar for a robot manipulator to form and re-arrange block patterns
in a table top setting. 
%
% Tenorth et al. \cite{tenorth2010understanding,lisca2015towards} use human activity demonstrations to train a relational probabilistic model to infer fully grounded symbolic plans from instructions. 
%
Misra et al. \cite{misra2016tell} present a similar approach but additionally assess plan feasibility for inferred plan candidates 
using a symbolic planner in the training loop. 
%
Despite successful demonstrations, these approaches require dense  
intermediate supervision and treat robot actions in a purely symbolic manner without learning their deep grounded semantics. 
%
% The proposed method learns a latent space of grounded programs which can be executed in the latent space of object representations. The explicitly reason with disentangled and modular concepts enables learning from \emph{only} the initial and final state data, forgoing the need for dense intermediate supervision. 
However, our method learns a latent space of composable symbolic programs which can be executed in the latent space of object representations. The composability of the  symbolic programs enables incremental learning through a specified curriculum from only the initial and final state data, forgoing the need for dense intermediate supervision. 


\textbf{Learning Action Models.} This work builds on the rich literature on acquiring action models for planning and captures \emph{when} actions are applicable and 
\emph{how} they affect the world. 
%
Efforts such as \cite{konidaris2018skills} and \cite{wang2021learning} 
learn initiation sets for actions implicitly learning to classify
the robot's configuration space where an action can be initiated. 
%
% Hannah et al. ~\cite{zettlemoyer2005learning} address the complementary problem of acquiring the 
% transition model and present an algorithm for determining \emph{planning rules} expressing 
% possible symbolic states resulting from the robot performing an action. 
%
~\cite{xia2018learning,silver2020few,zhu2021hierarchical} present neuro-symbolic approaches for learning a latent object-centric world model for tasks such as stacking, re-arrangement etc.  
%
In \cite{shridhar2022cliport}, authors introduce an end-to-end model that jointly predicts object affordances  and object displacements specifying metric goals for robot re-arrangement and sorting tasks. 
%
Recent parallel work by Wang et al. \cite{wang2023programmatically} builds on \cite{shridhar2022cliport} and represents the task as an executable program parsed from the instruction using a CCG parser, restricting the length and variation of the instructions it can handle. 
% 
Even though these approaches successfully learn modular action models that are amenable to planning, the scope of reasoning is limited to one-hop reasoning over spatial relations and attributes. In contrast, this work focuses on interpreting task instructions that require compositional/hierarchical spatial reasoning over an extended planning horizon. 



%% %%%
\iffalse

\textbf{Robot Skill Learning. }
The ability to reason and plan tasks is conditioned on the robot possessing knowledge 
of when it can perform actions and how it affects the world. 
%
Traditional approaches such as~\cite{knepper2013ikeabot} 
model robot actions as symbolic pre-conditions and effects. 
%
Such representations are often hand-coded making them 
brittle and error-prone in practice. 
%
Learning based efforts aim at acquiring actions representations through human demonstration 
or via self-supervision. 
%


\textbf{Robot Task Planning and Representations. }
Instead of focusing on learning primitive skills, complementary efforts have 
focused learning to translate a high-level instruction to a sequence of symbolic actions 
to accomplish multi-step tasks such as navigation or assembly. 

In ~\cite{paxton2019prospection}, authors introduce a recurrent architecture that 
translates language instructions into a sequence of sub-goals for the manipulator to follow. 
Central to the approach is a recurrent architecture that learns to predict sequential changes in the world 
model caused by robot actions. Although the model makes significant progress in predicting future world states, 
the model lacks the notion of grounded and executable representation for actions. As a result, 
the generalization performance degrades with tasks possessing repetitive structure that are unseen in training. 

The work in  ~\cite{paxton2019prospection} tackles the problem of learning spatial relational concepts 
and introduce a neural model that uses a bottleneck layer to recover disentangled representation for action. 
The task of converting an instruction to a sequence of symbolic actions is largely treated as sequence to 
sequence problem. Such a representation limits scope for rich reasoning and generalization to novel scenes.  
%
In contrast, this work adopts a program learning view, with explicit grounded representations for manipulation 
concepts that can be composed in symbolic programs allowing explicit reasoning with learned neural modules. 

In another effort, ~\cite{lazaro2019beyond}, authors introduce a cognitively-inspired approach for inferring 
executable programs. The authors demonstrate the ability to recover programs that the robot can execute to
modify block patterns to a target configuration. However, the authors assume a deterministic grounding from 
sensory inputs to discrete symbols. In contrast, this work builds an action representation directly on the 
perceived objects in the manipulation area. 


Finally, other efforts have attempted to learn rich logical description of tasks from human demonstration 
aimed towards inductive generalization to novel domains. 
%
 
 %
In contrast, this paper learns executable concepts that can be reasoned when combined in programs, 
endowing the ability to synthesize the world state after program execution.  
%
%Similarly, introduce a policy prior that effectively constrains the program space. 

\textbf{Instruction Following. }
Related efforts address the problem of inferring manipulation constraints \cite{howard2014natural, paul2016efficient} 
that that capture the intended goal from language utterance for the purposes of commanding a robot. 
%
Prominent models such as learn to associate linguistic constituents with a symbolic representation of the environment. 
%
The likely association or the \emph{grounding} for the instruction is used to derive a motion plan for the robot. 
%
Such approaches assume the presence of spatial concepts and focus on learning the association between language 
and they physical world state. 
%
The work presented in this paper, learns such spatial and action concepts directly from natural supervision forgoing the need 
for explicit hand coding of concepts. 
%
Other related efforts \cite{paul2018temporal, roy2019leveraging} 
learn groundings for actions and spatial relations assuming a set of hand-coded features 
or directly from data but lack an explicit representation of symbolic reasoning which may be needed for interpreting 
an instruction. 
%
The explicit notion of a program space used in this work makes learned grounded concepts amenable 
to symbolic reasoning.  
%

 
\textbf{Neuro-symbolic concept learning. } 
%
This paper builds on the Neuro-symbolic concept learning framework~\cite{Mao2019NeuroSymbolic}. 
The ability to acquire executable concepts from natural supervision. Applied to the visual question answering task. 
%
These approaches have successfully demonstrated the ability to learn static spatial concepts (relative positions, orientations)
and image attributes such as (appearance) etc. 
%
Recent efforts such as ~\cite{yi2019clevrer}, model dynamic interactions such as collisions an additionally model object consistency 
using temporally consistent appearance cues. 
%
We build on the model introduced in ~\cite{Mao2019NeuroSymbolic}. 
%
This work introduces the physical agent affecting the scene and 
address the task of learning grounded and executable representations for robot actions that cause changes in the 
environment in service of symbolic goals such as creating assemblies. 
%
More generally, neuro-symbolic approaches have also found applications in visual-question answering~\cite{yi2018neural}, 
modeling visual structure in scenes~\cite{li2020multi}, inferring motion programs~\cite{kulal2021hierarchical} etc.  
%
%To the best of our knowledge, the task of learning action representations for goal-directed physical agent affecting the environment 
%has not been addressed. 
% 

\fi

